%%%%%%%%%%%%%%%%%%%%%%%%%%%%%%%%%%%%%%%%%
% Press Release
% LaTeX Template
% Version 1.0 (2/6/13)
%
% License:
% CC BY-NC-SA 3.0 (http://creativecommons.org/licenses/by-nc-sa/3.0/)
%
%%%%%%%%%%%%%%%%%%%%%%%%%%%%%%%%%%%%%%%%%

%----------------------------------------------------------------------------------------
%	PACKAGES AND OTHER DOCUMENT CONFIGURATIONS
%----------------------------------------------------------------------------------------

\documentclass[11pt,pressrelease]{newlfm} % Font size

\usepackage{charter} % Use the Charter font for the document text

\PhrPhone{Phone} % Customize the "Telephone" text
\PhrEmail{Email} % Customize the "E-mail" text
%\PhrContact{Contact} % Uncomment this line to change the 'Contact:' text

%----------------------------------------------------------------------------------------
%	PRESS RELEASE INFORMATION
%----------------------------------------------------------------------------------------

\makeletterhead{Uiuc}{\Cheader{\vspace{16pt}\includegraphics[width=0.3\linewidth]{TSF_Logo_NB.jpg}}} % Include a company logo, if you don't use one you will need to uncomment line 6 in the prsrls.tex file
\lthUiuc % Print the company/institution logo

\release{For Immediate Release to Tartan Student Fund} % When the press release may be used

\namefrom{David Hiles, Head of Allocation} % Name


\phonefrom{732-796-8352} % Phone number

\emailfrom{dhiles@cmu.edu} % Email address

\headline{Summer Portfolio Updates} % Headline for the press release

\newcommand{\subtitle}{This is a subtitle to the headline giving more information about the headline.}  % Subtitle for the press release, if you don't want one just remove the subtitle text leaving the rest of the command

\byline{\textbf{Carnegie Mellon University, -- \today ~--} A review including outlooks for each Group.} % A summary line for the press release

%----------------------------------------------------------------------------------------

\begin{document}
\begin{newlfm}

%----------------------------------------------------------------------------------------
%	PRESS RELEASE CONTENT
%----------------------------------------------------------------------------------------

\vspace{-.25 in} 			%.65  is normal
\begin{singlespace} 		% Uncomment for single line spacing

\begin{enumerate}
\item \textbf{Consumer Discretionary} \par
As per our End of Semester (EOS) trades, we have pared down our position in \$NKE. We are good at buying stocks at their 52 week highs, so we are still struggling with this position. We are looking to see the results of \textbf{Nike}'s previous quarter, particularly anticipated revenue growth. Otherwise \$VDC has performed to expectations.

\item \textbf{Consumer Defensive} \par
This is the new group this semester and only has the (\textbf{Vanguard Consumer Staples ETF}, \$VDC). Our outlook is encouraging strong revenue with \underline{strong cash flow businesses.} We want to be careful of the Consumer Defensive sector's valuations, as markets have bought into this sector in the past 6 months. We have participated in most of the gains in this sector. 

\item \textbf{Healthcare} \par
Sticking with the theme of ``investors have preferred value over growth,'' our value healthcare plays have \textit{shown double digit positive returns} vs the S\&P while our \textit{growth segment has shown double digit losses.}  We will be rolling back our growth positions at attracticve prices and further allocating towards more attractive valuations. As our previous update included, ``We will closely monitor our \$TEVA position as it is involved with the souring Allergan-Pfizer deal.'' As many of you know this deal has ``soured'' and we saw a pretty substantial loss in the \textbf{TEVA} position. Our 23\% loss in this position is attributable to both the failed merger (a major catalyst) and a loss of appetite for risk in this sector. 

\item \textbf{Power \& Utilities} \par
The PU group has rebounded strongly through the summer, primarily due to a slight rebound and stabilization of oil and gas prices. EOS transition into \$RYE was a strong move, up ~15\% in a little over 5 months. We have seen \$PSX recover to a stable level, making back a large loss.  With respect to utilities, we find the ``Thirst for Yield'' outweighs the risk in a probable Fed Rate Hike this December. We will be encouraging cash producing, strong dividend businesses from this group moving forward. 

\item \textbf{Industrials \& Materials} \par
Our ``accidently passive'' strategy has proved fairly strong in this group.  Our ETF (\$XLB, \$VIS) positions have been resilient, with a slight retracement this September. We are looking for businesses that can capture some rebound in commodities prices, but considering currency fluctuations in their thesis. 

\item  \textbf{Tech, Media \& Telecom Group} \par
With the EOS purchase of AT\&T, we are beginning to scale back on the ETF's. We missed a 20\% exit opportunities in our \textbf{Telecom ETF} (\$VOX) 10\% gain, but we are still confident in the strength in this industry. We have maintained exposure to the \textbf{PowerShares Dynamic Media ETF}, which may be cut as it has proven to lack catalysts to move upwards with the broader tech group. After our purchase of AT\&T, the TMT group pitched a riskier play in cloud computing with Oracle. ORCL seems to be our loser in this group partly due to disappointing earnings (revenue down ~20\%, but only a linear decrease in EPS) but markets may react positively to the \$9 billion deal to buy NetSuite, announced in mid-2016.\par

%\vspace{12 pt}

\center \textbf{Summary} \par
\raggedright
We will be anticipating another quarter of ``Fat and Flat,'' implying higher volatility and smaller upside potential. With a strong earnings season, we might see some undervalued stocks realize gains, but most equities have a smaller range to advance given high valuations.I attached a chart to this memo that uses one of my favorite fundamental indicators, the The Advance-Decline Line (AD Line). It calculates ``Net Advances'', which is the number of advancing stocks less the number of declining stocks. Net Advances is positive when advances exceed declines and negative when declines exceed advances. As we can see, the amount of stocks advancing in the past month has declined ~10\%, so we expect the S\&P index to slow it's advance from the \textit{Post Brexit All-Time Highs}

\begin{center}
\fbox{\includegraphics[width=.85\linewidth]{NYAD_chart.pdf}}
\end{center}

\end{enumerate}
Please reach out if you have any questions or comments regarding the portfolio. 

Thank you, \par

Research, Risk and Strategy


\end{singlespace} % Uncomment for single line spacing


%----------------------------------------------------------------------------------------

\end{newlfm}
\end{document}