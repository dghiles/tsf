%%%%%%%%%%%%%%%%%%%%%%%%%%%%%%%%%%%%%%%%%
% Press Release
% LaTeX Template
% Version 1.0 (2/6/13)
%
% License:
% CC BY-NC-SA 3.0 (http://creativecommons.org/licenses/by-nc-sa/3.0/)
%
%%%%%%%%%%%%%%%%%%%%%%%%%%%%%%%%%%%%%%%%%

%----------------------------------------------------------------------------------------
%	PACKAGES AND OTHER DOCUMENT CONFIGURATIONS
%----------------------------------------------------------------------------------------

\documentclass[11pt,pressrelease]{newlfm} % Font size

\usepackage{etoolbox}
\makeatletter
\patchcmd{\@zfancyhead}{\fancy@reset}{\f@nch@reset}{}{}
\patchcmd{\@set@em@up}{\f@ncyolh}{\f@nch@olh}{}{}
\patchcmd{\@set@em@up}{\f@ncyolh}{\f@nch@olh}{}{}
\patchcmd{\@set@em@up}{\f@ncyorh}{\f@nch@orh}{}{}
\makeatother

\usepackage{charter} % Use the Charter font for the document text

\PhrPhone{Phone} % Customize the "Telephone" text
\PhrEmail{Email} % Customize the "E-mail" text
%\PhrContact{Contact} % Uncomment this line to change the 'Contact:' text

%----------------------------------------------------------------------------------------
%	PRESS RELEASE INFORMATION
%----------------------------------------------------------------------------------------

\makeletterhead{Uiuc}{\Cheader{\vspace{16pt}\includegraphics[width=0.3\linewidth]{TSF_Logo_NB.jpg}}} % Include a company logo, if you don't use one you will need to uncomment line 6 in the prsrls.tex file
\lthUiuc % Print the company/institution logo

\release{For Immediate Release to Tartan Student Fund} % When the press release may be used

\namefrom{David Hiles, Head of Equity Strategy Team} % Name


\phonefrom{732-796-8352} % Phone number

\emailfrom{dhiles@cmu.edu} % Email address

\headline{Summer Portfolio Updates} % Headline for the press release

\newcommand{\subtitle}{This is a subtitle to the headline giving more information about the headline.}  % Subtitle for the press release, if you don't want one just remove the subtitle text leaving the rest of the command

\byline{\textbf{Carnegie Mellon University, -- \today ~--} A review including outlooks for each Group.} % A summary line for the press release

%----------------------------------------------------------------------------------------

\begin{document}
\begin{newlfm}

%----------------------------------------------------------------------------------------
%	PRESS RELEASE CONTENT
%----------------------------------------------------------------------------------------

%\vspace{-.25 in} 			%.65  is normal
\begin{singlespace} 		% Uncomment for single line spacing
On some positive news, the TSF Portfolio is up 10.4\% since October 2016. However, the S\&P 500 is up a considerable amount more, showing a return around 20\%. Our portfolio has a smilair beta to the overall market, due to our similair sector allocation. 

\begin{enumerate}
\item \textbf{Consumer Discretionary \& Staples} \par
Our single name selections of Nike and KHC have proved to be a lag on the group, but carried by the ETF's the group has had solid prefromance over the year.
Our Nike postition has continuted to weigh on the portfolio despite the trim in late 2016 and has been rangebound for almost 18 months now. As per the broader investment themes, Nike's sales growth and projected sales growth has not kept up with the broader sector and investors have been cautious about ``Just doing it.'' As a group, I would recommend a revaluation of the KHC theses, parituclarly in the revenue growth catalysts. Both the Consumer Staples and Discretionary ETF’s have preformed well, although one would attribute this to the high correlations to the index. Our outlook moving foward varies per industry sector but with personal incomes rising (supplemented by record amounts of household credit), we may see further earnings growth in our consumer buisnesses. %Sensitivity to rates is a concern moving forward. Nike correlatuon with rates


\item \textbf{Healthcare} \par
Our two pharmactuial related postions Gilead and McKesson have not recovered to their previous highs due to pressure on drug prices. Gilead has recovered through the summer due to new delveopements in their storied cancer related treatments, but remains a ~20\% loss in the portfolio. Fundamentally, McKesson remains a strong player in the distibution space and I believe our thesis still holds. Like a handful of TSF picks in the past, we were incredibly skilled at buying McKesson at it's 2 year high and the postion is down 21\%. Our healthcare portfolio has been one of the hardest for us to manage, but our   current holdings are bouyed by our large postion in UNH (up 23\% since March) and the strong preformance of the ETF XLV.

\item \textbf{Power \& Utilities} \par
The PU group has rebounded strongly through the summer, primarily due to a slight rebound and stabilization of oil and gas prices. With supply-demand dyamnics in the crude markets stabliziling, managment teams have  CAPEX has been cut amongst many Exploration and Production companies. Thankfully we have been underweight E\&P for most of the fund's history. Our Midstream and Downstream positons in Spectra Energy Partners (Midstream MLP) and PSX (Refining and Marketing) have preformed well, but this is becoming a crowded trade . We have seen \$PSX recover to a stable level, making back a large loss. 

With respect to utilities, we find the bonb proxy theme has held through most of 2017. With real yeilds lower than expected, utilities with the highest (though often slower-growing) dividend yields soared to historical valuation peaks. Agian, we want cash producing, strong dividend businesses from this group moving forward. \$DUK has preformed inline with the broader ETF (\$XLU) up roughly 17\% YTD. Companies with lower dividend yields and higher earnings growth fell to attractive valuation levels.

\item \textbf{Industrials \& Materials} \par

The promise of infastructure spending has boosted these sectors to cycle highs. However, we are wary that this could be spectualtory inflows to the sector. The sector has out preformed and we have seen our ETF (\$XLB, \$VIS) positions up almost 30\% since last October. We are looking for businesses that are not dependent of rising commonidty prices as we do not have confidence in the commodity rebounding. This is mainly due to slower growth in commodity producing emerging markets and the effects currency impacts.

\item \textbf{Tech, Media \& Telecom Group} \par
With the EOS purchase of AT\&T, we are beginning to scale back on the ETF's. We have allocated away from Telcom in this portfolio, which has proven to be an effective move.

We missed a 20\% exit opportunities in our \textbf{Telecom ETF} (\$VOX) 10\% gain, but we are still confident in the strength in this industry. We have maintained exposure to the \textbf{PowerShares Dynamic Media ETF}, which may be cut as it has proven to lack catalysts to move upwards with the broader tech group. 

In recent develpoments, AT\&T missed Q3 analyst estimates and investors are worried about the lack of synergies with the DirectTV merger completed last year and mobile subscribers are shrinking. The stock has seen a volitile 19 months in our portfolio and we should be looking to reevaluate our thesis moving foward.

Like a few of our other portfolio groups, our clear winner in the TMT portfolio is the ETF \$IYW. It's slow advancement 

%\vspace{12 pt}

\center \textbf{Summary} \par
\raggedright
With current EPS for the S\&P 500 at around

We believe that this market could be pushed further given that low real yields to support higher than history steady-state PEs across the markets. The expectation of tax cuts and possible cyclical acceleration have weighed heavily on certain sectors, further driving returns.

We see some other themes in the value vs. growth tilt. The Russell 1000 Value Index has returned 9.7 percent in the first 10 months of 2017, while the Russell 1000 Growth Index has gained 24 percent. 

Looking at the fund's history, the biggest market moves have been in Technology and Consumer Stocks. Taking a page from Dalio's Economic model, we largely attribute this to unprecdented household credit growth, while maintining lower than average debt serivices levels. Takeaway? Without a widescale systematic event, we could continue to see this economic expansion continue past 2018 into 2020.

Besides sharp corrections in from December 2015 to January 2016 (-14\%), Technology and Consumer (both staples and discretionary) have followed (and driven) the wider trend of the bull market in the past 3-4 years. All other sectors have entered 



 Investors in the past year have been much more concerned with sales growth and forward sales growth. Many value funds have underpreformed the broader market, as investors look to buy disrupitve buisnesses,  at the expense of today's profits. We expect this trend to continue so long as yeilds stay low and thus the Equity Risk Premium can compensate investors.



With a strong earnings season, we might see some undervalued stocks realize gains, but most equities have a smaller range to advance given high valuations. I attached a chart to this memo that uses one of my favorite fundamental indicators, the The Advance-Decline Line (AD Line). It calculates ``Net Advances'', which is the number of advancing stocks less the number of declining stocks. Net Advances is positive when advances exceed declines and negative when declines exceed advances. As we can see, the amount of stocks advancing in the past month has declined ~10\%, so we expect the S\&P index to slow it's advance from the \textit{Post Brexit All-Time Highs}

\begin{center}
\fbox{\includegraphics[width=.85\linewidth]{NYAD_chart.pdf}}
\end{center}

\end{enumerate}
Please reach out if you have any questions or comments regarding the portfolio. 

Thank you, \par

Research, Risk and Strategy


\end{singlespace} % Uncomment for single line spacing


%----------------------------------------------------------------------------------------

\end{newlfm}
\end{document}